%-----------------------------------------------------------------%
% 2015 - 2023 Emerson Ribeiro de Mello - mello@ifsc.edu.br
% 
% 2023/04/10 versão 1.01
% 
% Modelo de monografia para o Instituto Federal de Santa Catarina
% 
% Adaptação do documento abnTeX2: Modelo de Trabalho Acadêmico
% para ficar de acordo com o "Template para elaboração de trabalho 
% acadêmico" fornecido pela Biblioteca do IFSC 
% https://www.ifsc.edu.br/documentos-uteis. Acesso em 2022-08-27
% 
%-----------------------------------------------------------------%
\documentclass[
	12pt,				% tamanho da fonte
	openright,			% capítulos começam em pág ímpar (insere página vazia caso preciso)
	oneside,			% para impressão em recto e verso. Oposto a oneside
	a4paper,			% tamanho do papel. 
	% -- opções da classe abntex2 --
	chapter=TITLE,		% títulos de capítulos convertidos em letras maiúsculas
	%english,			% idioma adicional para hifenização
	% french,				% idioma adicional para hifenização
	% spanish,			% idioma adicional para hifenização
	brazil				% o último idioma é o principal do documento
	]{ifscthesis}


%-----------------------------------------------%
% Capa
%-----------------------------------------------%

%-----------------------------------------------%
% Para alterar o gênero dos comandos orientador
% e coorientador.
%-----------------------------------------------%
% \renewcommand{\orientadorname}{Orientadora:}
%\renewcommand{\coorientadorname}{Coorientadora:}
%-----------------------------------------------%


%-----------------------------------------------%
% Informações de dados para CAPA e FOLHA DE ROSTO
%-----------------------------------------------%
\titulo{Análise Crítica \\
Aplicação da Ferramenta da Qualidade (Diagrama de Ishikawa) e do PDCA no Desenvolvimento de Pesquisa para a reutilização dos Resíduos Sólidos de Coco Verde
}
\autor{Lucas Alves de Lima}
\local{Jaraguá do Sul - SC}
\data{abril/2024}
\instituicao{
  Instituto Federal de Santa Catarin -- IFSC
  \par
  Campus Jaraguá do Sul - Rau
  \par
  Engenharia Elétrica}
\tipotrabalho{trabalho}
%-----------------------------------------------%




%-----------------------------------------------%
% O preambulo deve conter o tipo do trabalho, o objetivo, o nome da instituição e a área de concentração 
\preambulo{Trabalho apresentado pelos discentes do Curso de Engenharia Elétrica, 5ª Fase, como quesito avaliativo à disciplina de Fenômenos de Transporte do Instituto Federal de Santa Catarina - Campus Jaraguá do Sul - Rau e supervisionado pelo Professor Anderson José Antonietti.}

%\textoaprovacao{Este trabalho foi julgado adequado para obtenção do título de Engenheiro de Telecomunicações, pelo Instituto Federal de Educação, Ciência e Tecnologia de Santa Catarina, e aprovado na sua forma final pela comissão avaliadora abaixo indicada.}
%-----------------------------------------------%

%-----------------------------------------------%
% Estilo de cabeçalho que só contém o número da 
% página e uma linha
%-----------------------------------------------%
\makepagestyle{cabecalholimpo}
\makeevenhead{cabecalholimpo}{\thepage}{}{} % páginas pares
\makeoddhead{cabecalholimpo}{}{}{\thepage} % páginas ímpares
% \makeheadrule{cabecalholimpo}{\textwidth}{\normalrulethickness} % linha
%-----------------------------------------------%


%-----------------------------------------------%

%-----------------------------------------------%
% Incluir arquivo com acrônimos e símbolos
%-----------------------------------------------%
% 
% ATENÇÃO. Este projeto faz uso do pacote glossaries.
% Se for usar uma instalação local do LaTeX para
% compilar, e não no Overleaf, então é necessário 
% que tenha o arquivo .latexmkrc dentro diretório 
% deste projeto e use o comando abaixo:
% 
% latexmk -outdir=out -pdf monografia.tex
% 
% A extensão LaTeX Workshop do Visual Studio Code
% usa por padrão o latexmk para compilar
\makeglossaries


\newacronym
{json} % rótulo 
{JSON} % sigla
{\textit{JavaScript Object Notation}} % por extenso

\newacronym{ABNT}{ABNT}{Associação Brasileira de Normas Técnicas}

\newacronym{abnTeX}{abnTeX}{ABsurdas Normas para TeX}

\newacronym[longplural={Autoridades Certificadoras}]{AC}{AC}{Autoridade Certificadora}

\newacronym{AES}{AES}{\textit{Advanced Encryption Standard}}

\newacronym{TLS}{TLS}{\textit{Transport Layer Security}}

\newacronym{TPC}{TPC}{Terceira Parte Confiável}

\newacronym{IFSC}{IFSC}{Instituto Federal de Santa Catarina}
\glsxtrnewsymbol[description={conjunto vazio}]
{emptyset}% rótulo (será usado na ordenação na lista de símbolos)
{\ensuremath{\emptyset}}% símbolo


\glsxtrnewsymbol[description={número Pi}]
{pi}% rótulo (será usado na ordenação na lista de símbolos)
{\ensuremath{\pi}}% símbolo


%-----------------------------------------------%
% Início do documento
%-----------------------------------------------%
\begin{document}
% Seleciona o idioma do documento (conforme pacotes do babel)
\selectlanguage{brazil}




%-----------------------------------------------%
% ELEMENTOS PRÉ-TEXTUAIS
%-----------------------------------------------%
\pretextual
\imprimircapa
%-----------------------------------------------%

%-----------------------------------------------%
% No arquivo abaixo tem detalhes sobre folha de
% aprovação, ficha catalográfica, agradecimentos,
% resumo, abstract, etc.
% 
% Se não for a versão final do PDF, talvez fosse
% interessante comentar a linha abaixo.
%-----------------------------------------------%
% %-----------------------------------------------%
% Folha de rosto
% (o * indica que haverá a ficha bibliográfica)
%-----------------------------------------------%
% \imprimirfolhaderosto*
\imprimirfolhaderosto
%-----------------------------------------------%

%-----------------------------------------------%
% ficha bibliográfica
% 
% Pegue com a Biblioteca do IFSC um PDF com a 
% ficha correta, salve o arquivo no diretório
% deste projeto e descomente as linhas abaixo
% \begin{fichacatalografica}
%     \includepdf{ficha-catalografica.pdf}
% \end{fichacatalografica}
%-----------------------------------------------%

%-----------------------------------------------%


%-----------------------------------------------%
% folha de aprovação
%-----------------------------------------------%
%\begin{folhadeaprovacao}
%
%    \begin{center}
%        {\ABNTEXchapterfont\large\imprimirautor}
%
%        \vspace*{\fill}\vspace*{\fill}
%        \begin{center}
%            \ABNTEXchapterfont\Large\imprimirtitulo
%        \end{center}
%        \vspace*{\fill}
%
%        \imprimirtextoaprovacao
%
%        \vspace*{1cm}
%
%        \imprimirlocal, 10 de abril de 2023:
%
%        \vspace*{\fill}
%    \end{center}
%
%    % Alterando o espaço para assinatura de 0.7cm para 1.5cm
%    \setlength{\ABNTEXsignskip}{1.5cm}
%
%    \assinatura{\textbf{\imprimirorientador} \\ Orientador\\Instituto Federal de Santa Catarina}     
%    \assinatura{\textbf{Professor Fulano, Dr.} \\ Instituto Federal de Santa Catarina }
%    \assinatura{\textbf{Professora Fulana, Dra. } \\ Instituto Federal de Santa Catarina}
%    % \assinatura{\textbf{Professor Beltrano, Dr.} \\ Instituto Z}
%
%    \vspace*{1cm}
%  
%\end{folhadeaprovacao}
%-----------------------------------------------%


%-----------------------------------------------%
% Dedicatória
%-----------------------------------------------%
%\begin{dedicatoria}
%    \vspace*{\fill}
%    \begin{flushright}
%    \noindent
%    \textit{ Este trabalho é dedicado às crianças adultas que,\\
%    quando pequenas, sonharam em se tornar cientistas.}\vspace*{2cm}
%    \end{flushright}
% \end{dedicatoria}
%
%-----------------------------------------------%


%-----------------------------------------------%
% Agradecimentos
%-----------------------------------------------%
%\begin{agradecimentos}
%    Os agradecimentos principais são direcionados à Gerald Weber, Miguel Frasson,
%    Leslie H. Watter, Bruno Parente Lima, Flávio de Vasconcellos Corrêa, Otavio Real
%    Salvador, Renato Machnievscz\footnote{Os nomes dos integrantes do primeiro
%    projeto abn\TeX\ foram extraídos de
%    \url{http://codigolivre.org.br/projects/abntex/}} e todos aqueles que
%    contribuíram para que a produção de trabalhos acadêmicos conforme
%    as normas ABNT com \LaTeX\ fosse possível.
%    
%    Agradecimentos especiais são direcionados ao Centro de Pesquisa em Arquitetura
%    da Informação\footnote{\url{http://www.cpai.unb.br/}} da Universidade de
%    Brasília (CPAI), ao grupo de usuários
%    \emph{latex-br}\footnote{\url{http://groups.google.com/group/latex-br}} e aos
%    novos voluntários do grupo
%    \emph{\abnTeX}\footnote{\url{http://groups.google.com/group/abntex2} e
%    \url{http://www.abntex.net.br/}}~que contribuíram e que ainda
%    contribuirão para a evolução do \abnTeX.
%\end{agradecimentos}
%-----------------------------------------------%


%-----------------------------------------------%
% Epígrafe
%-----------------------------------------------%
%\begin{epigrafe}
%    \vspace*{\fill}
%    \begin{flushright}
%        \textit{Sempre que te perguntarem se podes fazer um trabalho,\\
%        respondas que sim e te ponhas em seguida a aprender como se faz.\\
%        T. Roosevelt}
%    \end{flushright}
%\end{epigrafe}
%-----------------------------------------------%


%-----------------------------------------------%
% Resumo e abstract
%-----------------------------------------------%
% ajusta o espaçamento dos parágrafos do resumo
%\setlength{\absparsep}{18pt} 
%\begin{resumo}
%    O resumo deve ressaltar o objetivo, o método, os resultados e as conclusões do documento. A ordem e a extensão destes itens dependem do tipo de resumo (informativo ou indicativo) e do tratamento que cada item recebe no documento original. O resumo deve ser precedido da referência do documento, com exceção do resumo inserido no próprio documento. O resumo deve conter apenas um parágrafo com no mínimo 150 e no máximo 250 palavras. As palavras-chave devem figurar logo abaixo do resumo, antecedidas da expressão Palavras-chave:, separadas entre si por ponto e finalizadas também por ponto. Este documento segue as normas da \gls{ABNT} e para isso faz uso do pacote \gls{abnTeX}.
%    
%    \textbf{Palavras-chave}: latex. abntex. editoração de texto.
%\end{resumo}

%-----------------------------------------------%
%\begin{resumo}[Abstract]
%\begin{otherlanguage*}{english}
%    This is the english abstract.
%\vspace{\onelineskip}
%
%\noindent 
%\textbf{Keywords}: latex. abntex. text editoration.
%\end{otherlanguage*}
%\end{resumo}
%-----------------------------------------------%
%-----------------------------------------------%

%-----------------------------------------------%
% Listas ilustrações, tabelas, códigos, abreviaturas
% símbolos.
% Comente a linha abaixo para não gerar as listas
%-----------------------------------------------%
%
%-----------------------------------------------%
% Listas ilustrações, tabelas, códigos, abreviaturas
% símbolos
%-----------------------------------------------%

% Lista de ilustrações
\pdfbookmark[0]{\listfigurename}{lof}
\listoffigures*
\cleardoublepage

% Lista de quadros
\pdfbookmark[0]{\listofquadrosname}{loq}
\listofquadros*
\cleardoublepage

% Lista de tabelas
\pdfbookmark[0]{\listtablename}{lot}
\listoftables*
\cleardoublepage

% Lista de códigos
\pdfbookmark[0]{\lstlistlistingname}{lol}
\begin{KeepFromToc}
\lstlistoflistings
\end{KeepFromToc}
\cleardoublepage

% Lista de abreviaturas
\printglossary[type=\acronymtype,nonumberlist,title=Lista de abreviaturas e siglas]
\cleardoublepage

% Lista de símbolos
\printglossary[type=symbols,nonumberlist,title=Lista de símbolos]
\cleardoublepage

% Sumário
\pdfbookmark[0]{\contentsname}{toc}
\tableofcontents*
\cleardoublepage
%-----------------------------------------------%


%-----------------------------------------------%
% Elementos textuais - Capítulos
%-----------------------------------------------%
% Se quiser que apareça o título dos capítulos
% no cabeçalho, então comente a linha abaixo
\pagestyle{cabecalholimpo}

%\input{superCap/superCap.tex}

\chapter{Introdução}\label{cap:introducao}

Este relatório tem como objetivo principal apresentar uma análise numérica dos fenômenos de transferência de calor em aletas. Estas superfícies são consideradas como uma extensão de uma base condutora, potencializando o efeito da dissipação de calor.

O estudo se concentra em aletas planas unidimensionais e tem como objetivo determinar os principais fatores que influenciam a transferência de calor nesses tipos de superfícies. A análise será feita por meio de simulações numéricas que permitirão quantificar a transferência de calor e avaliar a eficiência das aletas em dissipar o calor de forma adequada.

\section{Objetivos}

Dentro do escopo do objetivo principal, foram definidos alguns objetivos específicos:

\begin{itemize}[leftmargin=2cm]
   \item Determinar a temperatura da superfície da base da aleta (Tb);

   \item Determinar a taxa de transferência de calor de cada tipo de aleta (qa);

   \item Determinar a temperatura na ponta da aleta;

   \item Determinar a efetividade ({\large \(\epsilon\)}) da superfície estendida. Caso a aleta apresente ({\large\(\epsilon\)}1>2);

   \item Determinar a eficiência da aleta ({\large\(\eta\)}a);

   \item Determinar a eficiência global do conjunto aletado ({\large\({\eta}\)}o).
\end{itemize}




%-----------------------------------------------%
% ELEMENTOS PÓS-TEXTUAIS
%-----------------------------------------------%
\postextual
% %-----------------------------------------------%
%-----------------------------------------------%
% Referências bibliográficas
%-----------------------------------------------%
\bibliography{referencias}


%-----------------------------------------------%
% Apêndices
%-----------------------------------------------%
%\begin{apendicesenv}
%% Imprime uma página indicando o início dos apêndices
%\partapendices
%
%\chapter{Meu primeiro apêndice}
%
%Texto ou documento, elaborado pelo autor, a fim de complementar sua argumentação, sem prejuízo da unidade nuclear do trabalho. Os apêndices são identificados por letras maiúsculas ordenadas alfabeticamente, travessão e pelo respectivo título. 
%
%\end{apendicesenv}

%-----------------------------------------------%
% Anexos
%-----------------------------------------------%
%\begin{anexosenv}
%% Imprime uma página indicando o início dos anexos
%\partanexos
%
%\chapter{Meu primeiro assunto de anexo}
%
%Texto ou documento não elaborado pelo autor, que serve de fundamentação, comprovação e ilustração. Os anexos são identificados por letras maiúsculas ordenadas alfabeticamente, travessões e pelos respectivos títulos. 
%
%
%\chapter{Segundo assunto que pesquisei}
%\lipsum[31]
%
%\end{anexosenv}


    
\end{document}

